\documentclass[twoside,numberorder]{csbachelor}

%==============================================================
%==============================================================

\usepackage{url}
\usepackage{subfigure}

% 张海:其他引用
\usepackage[hidelinks]{hyperref}
\setlength{\LTpre}{1em}
\setlength{\LTpost}{1em}

\usepackage{tikz}
\usetikzlibrary{arrows,backgrounds,fit,shapes}
\tikzstyle{layer} = [draw, dashed]
\tikzstyle{block} = [draw, rectangle, minimum height=2em]
\tikzset{>=latex}

% 一些全局工具的定义
\DeclareMathOperator*{\argmin}{arg\,min}
\DeclareMathOperator*{\argmax}{arg\,max}

\def\E{{\rm E}}
\def\N{{\rm N}}
\def\KL{{\rm KL}}

\def\L{{\bf L}}
\def\M{{\cal M}}

\def\P{P_{\rm data}}

\def\D{{\cal D}}
\def\G{{\cal G}}

\def\tY{\tilde{Y}}
\def\tn{\tilde{n}}

\def\hY{\hat{Y}}
\def\hX{\hat{X}}

% For CoopNets paper formula
\usepackage{tikz}
\usepackage{algorithm}
\usepackage{algorithmic}
\usepackage{multicol,lipsum}
\usepackage{float}
\usepackage{color}
\usepackage{makecell}
% For table
\usepackage{tabularx}
% For toc (table of content)
\usepackage{subfigure}
\usepackage[subfigure]{tocloft}
\usepackage{hyperref}
% For toc dot line appear again after using tocloft
\renewcommand{\cftchapleader}{\cftdotfill{\cftdotsep}}
% not show num for part
\cftpagenumbersoff{part}
% For changing the name of part
\usepackage{xeCJK}
\usepackage{zhnumber}
% For include kaitibaogao
\usepackage{pdfpages}

% For header
%\usepackage{fancyhdr}
%\pagestyle{fancy}
%\fancyhead[LO]{\small 浙江大学本科生毕业论文}
%\fancyhead[RE]{\small 基于对比散度学习多尺度生成式卷积网络}
%\fancyfoot[C]{\small{\thepage}}

%==============================================================
%==============================================================

\begin{document}

%==============================================================
%==============================================================

%  % 论文题目:{中文}{英文}
%  \zjutitle{基于对比散度学习多尺度生成式卷积网络}
%           {Title}
%  % 作者:{中文姓名}{英文}{学号}
%  \zjuauthor{姓名}{Name}{313XXXXXXXX}
%  % 指导教师:{导师中文名}{导师英文名}
%  \zjumentor{导师姓名}{Supervisor name}
%  % 个人信息:{年级}{专业名称}
%  \zjuinfo{2013级}{计算机科学与技术}{Computer Science and Technology}
%  % 学院信息:{学院中文}{学院英文}
%  \zjucollege{计算机科学与技术学院}{College of Computer Science and Technology}
%  % 日期:{提交日期}{Submitted Date}
%  \zjudate{2017 年 6 月 5 日}{June 5, 2017}

%==============================================================

%  {
%    \pagestyle{empty}
%
%    \include{data/cover-zh}
%    \cleardoublepage
%    \include{data/cover-en}
%    \cleardoublepage
%
%  }

  {
    \frontmatter

    \pagestyle{frontmatter}
    \makeatletter
      \let\ps@plain\ps@frontmatter
    \makeatother

	{
  \setlength{\parindent}{0em}
  \linespread{1}

  \vspace*{0.6em}

  {
    \centering
    \fangsong\sanhao
    浙江大学本科生毕业论文承诺书 \par
  }

  \vspace{3.1em}

  {
    \setlength{\parindent}{2em}
    \linespread{1.6}
    \fangsong\sihao
    1. 本人郑重地承诺所呈交的毕业论文,是在指导教师的指导下严格按照学校和学院有关规定完成的。
    
    2. 本人在毕业论文(设计)中除了文中特别加以标注和致谢的地方外,论文中不包含其他人已经发表或撰写过的研究成果,也不包含为获得 \underline{\fangsong\sihao\bfseries \makebox[5em]{浙江大学}} 或其他教育机构的学位或证书而使用过的材料。
    
    3. 与我一同工作的同志对本研究所做的任何贡献均已在论文中作了明确的说明并表示谢意。
    
    4. 本人承诺在毕业论文工作过程中没有伪造数据等行为。
    
    5. 若在本毕业论文(设计)中有侵犯任何方面知识产权的行为,由本人承担相应的法律责任。
    
    6. 本人完全了解 \underline{\fangsong\sihao\bfseries \makebox[5em]{浙江大学}} 有权保留并向有关部门或机构送交本论文的复印件和磁盘,允许本论文被查阅和借阅。本人授权 \underline{\fangsong\sihao\bfseries \makebox[5em]{浙江大学}} 可以将本论文(设计)的全部或部分内容编入有关数据库进行检索和传播,可以采用影印、缩印或扫描等复制手段保存、汇编本论文。
     \par
  }

  \vspace{2.9em}

  {
	\fangsong\sihao
	\begin{tabular}{@{} p{0.5\linewidth} p{0.5\linewidth} @{}}
		作者签名: & 导师签名: \\
		& \\
		& \\
		签字日期: \hspace{2em} 年 \hspace{2em} 月 \hspace{2em} 日 & 签字日期: \hspace{2em} 年 \hspace{2em} 月 \hspace{2em} 日 \\
	\end{tabular} \par
	}
%
%  \vspace{4.85em}
%
%  {
%    \centering
%    \fangsong\xiaoer
%    毕业论文(设计)版权使用授权书 \par
%  }
%
%  \vspace{2.2em}
%
%  {
%    \setlength{\parindent}{2em}
%    \linespread{1.6}
%    \fangsong\xiaosi
%    本文作者完全了解 \underline{\kaiti\sihao\bfseries \makebox[5em]{浙江大学}} 有权保留并向国家有关部门或机构送交本文的复印件和磁盘,允许本文被查阅和借阅。本人授权 \underline{\kaiti\sihao\bfseries \makebox[5em]{浙江大学}} 可以将毕业论文(设计)的全部或部分内容编入有关数据库进行检索和传播,可以采用影印、缩印或扫描等复制手段保存、汇编毕业论文(设计)。
%
%    (保密的毕业论文(设计)在解密后适用本授权书) \par
%  }
%
%  \vspace{2.9em}
%
%  {
%    \songti\xiaosi
%    \begin{tabular}{@{} p{0.5\linewidth} p{0.5\linewidth} @{}}
%    作者签名: & 导师签名: \\
%     & \\
%     & \\
%    日期: \hspace{4em} 年 \hspace{2em} 月 \hspace{2em} 日 & 日期: \hspace{4em} 年 \hspace{2em} 月 \hspace{2em} 日 \\
%    \end{tabular} \par
%  }
}

	%\cleardoublepage

    \chapter*{致谢}
%\addcontentsline{toc}{chapter}{致谢}
\markright{致谢}
%\linespread{1.25}
%\fangsong\xiaosi

首先要感谢的是


%\cleardoublepage

    \chapter*{摘要}

由于世界上大部分的原始数据都不带标签

{
    \vspace{1em}
    \setlength{\parindent}{0em}
    \textbf{关键词} \; 无监督学习 \; 生成式模型 \; 卷积神经网络 \; 多尺度 \; 对比散度 \; 图像生成 \par
}

    \chapter*{Abstract}
\timesnewroman

Most data in the world don't have labels, and labeling data by humans is too expensive, so the unsupervised learning receives more and more attentions. 

{
    \vspace{1em}
    \setlength{\parindent}{0em}
    \textbf{Keywords} \; Unsupervised Learning \; Generative Model \; Convolutional Neural Networks \; Multi-grid Method \; Contrastive Divergence \; Image Generation \; \par
}

	
    \tableofcontents
    \newpage
    \thispagestyle{empty}
    
  }

	\pagenumbering{gobble}
	\part{毕业论文}
	\pagenumbering{arabic}

  {
    \mainmatter

    \pagestyle{mainmatter}
    \makeatletter
      \let\ps@plain\ps@mainmatter
    \makeatother

    \chapter{绪论}

\section{课题背景}

现在,在很多领域,机器学习已经成为了主流的方法\cite{nasrabadi2007pattern},


\section{本文研究目标和内容}

训练生成式模型有多种方法,


%\begin{figure}[!htbp]
%\centering
%\includegraphics[width=\linewidth,keepaspectratio]{data/chapter-1/placeholder.png}
%\caption{示例图片}
%\label{figure:sample}
%\end{figure}
%
%\section{本文结构安排}
%
%如表~\ref{table:sample}
%
%\begin{table}[!htbp]
%\caption{示例表格}
%\label{table:sample}
%\centering
%\begin{tabular}{|c|c|}
%\hline
%键 & 值 \\
%\hline
%键 1 & 值 1 \\
%\hline
%\end{tabular}
%\end{table}

    %\cleardoublepage
    \chapter{文献综述/技术路线}

\section{基于能量的生成式卷积网络}

因为本文主要关注生成式模型中的基于能量生成式卷积网络,所以这一节将会简单介绍基于能量的生成式卷积网络的模型结构,并且解释其和判别式网络的等价性。

\subsection{对数倾斜}
假设$Y$是一张正方形(或者矩形)的图片



    %\cleardoublepage
    \chapter{研究方案/我做的工作}

\section{公式推导} \label{formulaDeduction}

我们选用的卷积网络模型与Xie在2016年的工作\cite{xie2016cooperative}里的descriptor网络十分接近,

    %\cleardoublepage
    \chapter{实验结果}

\section{图像生成实验结果} \label{SynthesizeRes}

在 CelebA 数据集中,我们随机采样了 10,000 张图片用来训练。



    %\cleardoublepage
    \chapter{结论}

这篇文章提出了一种多尺度的模型,可以训练基于能量的生成式卷积网络,并且不需要借助于类似 GAN 中的生成器结构。




    %\cleardoublepage
    
    \renewcommand{\thechapter}{}
    \include{data/bibliography}
    
    \chapternonum{作者简历}

\vspace{2cm}

{
	\setlength{\parindent}{0em}
	
	\textbf{姓名}:~周君沛 
	
	\textbf{性别}:~xxx
	
	\textbf{民族}:~xxx
	
	\textbf{出生年月}:~xxx
	
	\textbf{籍贯}:~xxx
	
	\vspace{1cm}
	
	2011.09 - 2014.07 ~~xxx
	
	2014.09 - 2018.07 ~~浙江大学攻读学士学位
	
	\vspace{1cm}
	
	\textbf{获奖情况}:~xxx奖学金
	
	\textbf{参加项目}:~xxx
	
	\textbf{发表的学术论文}:~xxx

}



    \chapternonum{浙江大学本科生毕业论文任务书}

{
	\setlength{\parindent}{0em}
	\renewcommand{\baselinestretch}{2}
	\fangsong\xiaosi\bfseries
	
	一、 \; 题目: \; 基于对比散度学习多尺度生成式卷积网络
	\vspace{2em}

	二、 \; 指导教师对毕业论文的进度安排及任务要求:
	\vspace{2em}
	
}	

{
	\fangsong\xiaosi
	在进度安排方面,建议从七月份开始就确定好题目,并且阅读计算机视觉相关论文,学习统计学相关的概念。随后可以阅读生成式模型的相关论文,对各种生成式模型形成了解,然后针对之前的模型不足的地方思考解决办法,并且调研各种公开数据集。九月份开始可以阅读论文了解评价生成式模型的各种指标以及目前的最高水平,随后十月份可以熟悉常用的一些深度学习框架并复现各种基线算法及其他论文。十一月份建议开始熟悉编程将要用的框架,实现自己对模型的改进,随后改进模型结构,优化模型参数,训练深度模型,接着在年底跑各种实验并且记录实验结果,用相同实验任务测试其他模型。
}

	\vspace{2cm}

{
	\setlength{\parindent}{0em}
	\fangsong\xiaosi\bfseries
	
	起讫日期 ~~ 2017 年  7 月 1 日 至 2018 年 5 月 1 日
	
	\begin{flushright}
		指导教师(签名) \; \underline{\hspace{6em}} ~~~~ 职称 \; \underline{\hspace{3em}}\\
		年 \qquad 月 \qquad 日
	\end{flushright}
}


{
	\setlength{\parindent}{0em}
	\renewcommand{\baselinestretch}{2}
	\fangsong\xiaosi\bfseries
	
	三、 \; 系或研究所审核意见:
	
	\vspace{2cm}
}

{
	\fangsong\xiaosi\bfseries
	
	\begin{flushright}
	负责人(签名) \; \underline{\hspace{6em}} \\
	年 \qquad 月 \qquad 日
	\end{flushright}
}

\newpage

\chapternonum{浙江大学本科生毕业论文考核表}

{
	\setlength{\parindent}{0em}
	\renewcommand{\baselinestretch}{2}
	\fangsong\xiaosi\bfseries
	
	一、 \; 指导教师对毕业论文的评语:
	\vspace{8em}
	
	\begin{flushright}
		指导教师(签名) \; \underline{\hspace{6em}} \\
		年 \qquad 月 \qquad 日
	\end{flushright}

	\vspace{1em}

	二、 \; 答辩小组对毕业论文(设计)的答辩评语及总评成绩:
	\vspace{8em}
	
	\begin{table}[H]
		\centering \bfseries \wuhao
		\begin{tabularx}{\textwidth}{|>{\fangsong}c
				|>{\fangsong}X<{\centering}
				|>{\fangsong}X<{\centering}
				|>{\fangsong}X<{\centering}
				|>{\fangsong}X<{\centering}
				|>{\fangsong}c|}
			\hline
			\makecell{成绩\\比例} & \makecell{文献综述\\(10\%)}& \makecell{开题报告\\(15\%)} & \makecell{外文翻译\\(5\%)} & \makecell{毕业论文\\质量及答辩\\(70\%)} & \makecell{总评成绩} \\
			\hline
			分值 &  &  &  &  &  \\
			~ & ~ & ~ & ~ & ~ & ~ \\
			\hline
		\end{tabularx}
	\end{table}
	
	\begin{flushright}
	答辩小组负责人(签名) \; \underline{\hspace{6em}} \\
	年 \qquad 月 \qquad 日
	\end{flushright}
	
}	

    
  }

	\pagenumbering{gobble}
	\part{文献综述和开题报告}
	\pagenumbering{arabic}
	
	\includepdf[pages=-]{kaitibaogao.pdf}

%  {
%    \backmatter
%
%    \pagestyle{empty}
%
%    \include{data/assignment}
%    \cleardoublepage
%
%    \include{data/assessment}
%    \cleardoublepage
%  }

\end{document}
